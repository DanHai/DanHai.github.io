%%%%%%%%%%%%%%%%%%%%%%%%%%%%%%%%%%%%%%%%%
% Medium Length Professional CV
% LaTeX Template
% Version 2.0 (8/5/13)
%
% This template has been downloaded from:
% http://www.LaTeXTemplates.com
%
% Original author:
% Rishi Shah 
%
% Important note:
% This template requires the resume.cls file to be in the same directory as the
% .tex file. The resume.cls file provides the resume style used for structuring the
% document.
%
%%%%%%%%%%%%%%%%%%%%%%%%%%%%%%%%%%%%%%%%%

%----------------------------------------------------------------------------------------
%	PACKAGES AND OTHER DOCUMENT CONFIGURATIONS
%----------------------------------------------------------------------------------------

\documentclass{resume} % Use the custom resume.cls style
\usepackage{hyperref}

\usepackage[left=0.75in,top=0.6in,right=0.75in,bottom=0.6in]{geometry} % Document margins
\newcommand{\tab}[1]{\hspace{.2667\textwidth}\rlap{#1}}
\newcommand{\itab}[1]{\hspace{0em}\rlap{#1}}

\name{Hai Dang} % Your name
\address{Doctoral HCI Researcher} % Your address
\address{h.dang@uni-bayreuth.de} % Your phone number and email

\begin{document}

I'm a Ph.D. candidate at the university of Bayreuth and part of 
the \href{https://www.hciai.uni-bayreuth.de/en/index.html}{HCI+AI research group}.
My work focuses on conceptualizing and building new tools to enable non-expert users to continuously
interact with generative models and craft new things with the help of deep learning models.

%----------------------------------------------------------------------------------------
%	EDUCATION SECTION
%----------------------------------------------------------------------------------------

\begin{rSection}{Education}
{\bf University of Bayreuth, Bayreuth Germany } \hfill {\em 2020/09 - 2024/07} 
\\ Doctor of Philosophy
\\ Department of Computer Science \\
\\
{\bf LMU Munich, Munich Germany } \hfill {\em 2018/10 - 2020/07} 
\\ Master of Science
\\ Department of Computer Science \hfill { GPA: 3.7 | (German Scale: 1.23) }
\\ Thesis: Representational Learning for Exploring\\Input Spaces in HCI\\
\\{\bf LMU Munich, Munich Germany} \hfill {\em 2013/10 - 2018/07} 
\\ Bachelor of Science
\\ Department of Computer Science \hfill { GPA: 3.7 | (German Scale: 1.38)}
\\ Thesis: Deep Conformance Checking \\ Efficient Estimation of Alignment Based Fitness.\\
\\{\bf Yonsei University, Seoul South Korea} \hfill {\em 2016/08 - 2017/07} 
\\ Bachelor of Science
\\ Department of Computer Science \hfill {\em Year Abroad} 

\end{rSection}

%----------------------------------------------------------------------------------------
%	TECHNICAL STRENGTHS SECTION
%----------------------------------------------------------------------------------------

\begin{rSection}{Technical Strengths}

    \begin{tabular}{ @{} >{\bfseries}l @{\hspace{6ex}} l }
    Python Libraries    \ & PyTorch (primarily), Pandas, Numpy, Tensorflow\\
    Programming Languages: \ &  Python (primarily), JavaScript / Typescript, Java\\
    Web-Frameworks:            \ &  React, D3js, Flask / FastAPI \\
    DevOps:     \ & Docker, Nginx, Git \\
    OS:  \ & Unix Systems (primarily), Windows (occasionally)\\
    \end{tabular}
    
\end{rSection}

\begin{rSection}{Publication}
    Hai Dang and Daniel Buschek. 2021. \textbf{GestureMap: Supporting Visual Analytics and Quantitative Analysis of Motion Elicitation Data by Learning 2D Embeddings.}
    In \textit{Proceedings of the 2021 CHI Conference on Human Factors in Computing Systems} (\textit{CHI '21}). Association for Computing Machinery, New York, NY, USA, Article 317, 1–12. DOI: \url{https://doi.org/10.1145/3411764.3445765}
\end{rSection}

%--------------------------------------------------------------------------------
%    Projects And Seminars
%-----------------------------------------------------------------------------------------------
\begin{rSection}{University Projects}

\begin{rSubsection}{Representational Learning for Exploring Input Spaces in HCI}{SS20}{Master Thesis}{}
\item Developed an interactive tool to analyze and visualize gesture elicitation studies in HCI.
\item Used a VAE to learn a two-dimensional gesture space from depth images recorded with a Kinect Sensor.
\item Conducted expert interviews to evaluate the tool.
\end{rSubsection}

\vfill

\begin{rSubsection}{Ensemble Knowledge Graph Embedding}{WS 2019}{Seminar: Group Project}{}
% \item 
\item Reimplemented ConvE (Dettmers et al. 2018) and improved computation time by 30\% by using adequate vectorization.
\item Managed and monitored the machine learning models and the remote training clouds using MlFlow and DVC.
\end{rSubsection}

% \begin{rSubsection}{Sketching with Hardware}{Summer Semester 2019}{Workshop: Group Project}{}
% \item Topic: Design an interactive game.
% \item Build an interactive beer-pong platform named Pongly.
% \item The game consisted of two wooden rotating platforms, each driven by an electric motor. The speed of the electric motor was controlled by a potentiometer, which players could manipulate during the game to change the rotation speed. Each platform could hold six drinking cups. A LED light and a magnetic sensor in each slot were used to track and visualize whether a cup was placed in the slot. The color of the LED lights indicated which game mode was currently active, e.g., rapid-fire, bonus points, locked cups, and so on.
% \item Programmed the game logic into an Arduino board that was attached to the platforms.
% \item Worked with the laser cutter, 3D printer, and other fabrication tools to produce the components for the game.
% \end{rSubsection}

\begin{rSubsection}{Evaluation of Consumer Grade BCI Devices}{SS 2019}{Seminar: Group Project}{}
% \item Conducted multiple experiments with a consumer-grade EEG-Cap measuring the cognitive workload level of individual study participants using the N-Back task.
\item Applied basic signal processing techniques on the raw EEG recordings to extract the alpha and theta frequencies that characterize the cognitive workload.
\item Trained multiple classifiers from the SciKit library to differentiate between various workload levels.
\end{rSubsection}

\begin{rSubsection}{Development of an Interactive Sleep Monitoring Device}{SS 2019}{Seminar: Group Project}{}
% \item Goal: Conceptualized and developed an interactive sleep monitoring device.
\item Built the analytics backend to collect and analyze sleep data.
\item Designed the communication protocol between the device and the analytics backend.
\end{rSubsection}

\begin{rSubsection}{Power Efficient High Performance Computing}{WS 2018}{Seminar: Group Project}{}
% \item Goal: Develop a machine learning model to predict the power consumption of a data center in Munich.
% \item Studied the center's warm-water cooling system.
\item Developed a recurrent neural network model for the prediction of energy consumption.
\item Achievement: Won the class competition for most accurate predictions by employing an autoregressive recurrent neural network.
\end{rSubsection}

% \begin{rSubsection}{Intelligent User Interfaces}{Winter Semester 2018}{Seminar: Group Project}{}
% \item Goal: Develop various intelligent system prototypes.
% \item Developed an Amazon Alexa voice application to navigate our universities' extracurricular activities web page. It was aimed at first-year undergraduate students who wanted to explore the offerings of our university. 
% \item Developed a basic text summarization and auto-completion tool based on the NLTK library.
% \item Experimented with the IBM Watson cloud to build a system that identifies craft beer bottles and recommends a matching sausage type.
% \end{rSubsection}

\end{rSection}

%\break

%----------------------------------------------------------------------------------------
%	WORK EXPERIENCE SECTION
%----------------------------------------------------------------------------------------

\begin{rSection}{Selected Work Experience}

\begin{rSubsection}{SWM, Munich}{2019/08 - 2019/11}{Machine Learning Developer}{Working Student}
\item Collected and integrated electricity data from various 
transmission system operators
\item Implemented multiple autoregressive models to predict feed-in-management operations
\item Built and deployed an end-to-end machine learning solution
\end{rSubsection}

\begin{rSubsection}{Celonis, Munich}{2017/10 - 2019/02}{Software Developer / Data Scientist}{Working Student}
\item Responsible for the Python Data Push API for the Celonis Business Intelligence Cloud Platform.
\item Developed a pipeline to programmatically analyze event data.
\item Developed business analyses and gave data science workshops
\item Developed tools to automate and administrate the Celonis Process Mining Platform
\end{rSubsection}

% \begin{rSubsection}{LMU Munich, Munich}{April 2016 - August 2016}{Tutor - Multi-Media-Programming}{Working Student}
% \item Created and presented material for the lecture
% \item Assisted the course participants during their programming assignments
% \end{rSubsection}

% \begin{rSubsection}{FeldM, Munich}{April 2015 - April 2016}{Java Software Developer}{Working Student}
% \item Developed a tool to monitor and evaluate different social media channels
% \item Extended an internal management tool to track working hours
% \end{rSubsection}

\end{rSection}
%----------------------------------------------------------------------------------------
% Course Work
%----------------------------------------------------------------------------------------
\begin{rSection}{Selected Relevant Coursework} \itemsep -3pt
{\bf Data Processing and Analysis:} Deep Learning Algorithms, Machine Learning, Knowledge Discovery in Databases, Probability and Measure Theory, Big Data Management\\
{\bf Human-Computer-Interaction:} Advanced topics on HCI, Intelligent User Interfaces, Human-Computer-Interaction\\
{\bf General Software Development:} Data Structures and Algorithms, Software Architecture, Software Testing, Agile Development
\end{rSection}

\end{document}
